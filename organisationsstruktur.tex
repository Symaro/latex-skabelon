\section{Spørgsmål 2 - Fleksibel Organisationsstruktur}
formålet for denne opgave at belyse de fordele og ulemper der kan være forbundet med en agil organisationsstruktur.
\\Dette gøres med afsæt i opgivet pensum samt de resterende to artikler opgivet i eksamensopgaven, og vil blivere refereret til på følgende måde
\begin{itemize}
    \item \cite[a.4]{eksamensopgave} - "Konsulentfirma vil skabe fremtidens organisationer"
    \item \cite[a.5]{eksamensopgave} - "Tænk over de signaler din organisation sender til verden"
\end{itemize}
\subsection{Organisationsstrukturtyper}
Organisationstrukturen kan opdeles i top typer: mekanistisk og organisk
\paragraph{Mekanistisk struktur}
\begin{itemize}
    \item Faste og låste roller
    \item Centraliserede beslutningsprocesser
    \item Funktionel separering
    \item Standardiseret
    \item Vertikalt kommunikationsflow
    \item Bureaukratisk
\end{itemize}
\paragraph{Organisk struktur}
\begin{itemize}
    \item Fleksible, tangerende til flydende, roller
    \item Decentraliserede beslutningsprocesser
    \item Integreret funktionalitet
    \item Gensidig tilvænning og kompromiser
\end{itemize}
Begge artikler lægger vægt på at den organiske struktur er overlegen ift. den mekanistiske.
\\Det er rimeligt at antage at en organisk struktur kan være at foretrække når man snakker om organisationer som befinder sig på komplekse markeder i evig forandring,
dog skal man huske at dette ikke er tilfældet for alle organisationer.
\\Den mekanistiske struktur er fordelagtig når det ikke er nødvendigt for organisation at tilpasse sig et skiftende miljø, eller når autoritetskæden er et fokuspunkt.
Hvilket typisk kan ses indenfor de forskellige militære faktioner og autoritetsinstancer.
\\Den organiske struktur egner sig til organisationer, hvor tilpasning til miljøændringer og arbejdsopgaver er i højsædet. Den giver medarbejderne mere frihed til at arbejde på tværs af fagområder,
hvorved de opbygger flere kompetencer, hvilket også kan føre til en øget produktivitet \cite[s. 132]{jones:2013}
\\~\\Umiddelbart vil organisationer ofte kunne drage fordel af at indkorperere en organisk struktur, især hvis de arbejder med produkter hvor livcyklussen er relativ kort eller hastigheden hvormed teknologien ændrer sig er høj.
Hvorfor en hurtig omstilling er essentiel.
\\Det kan dog være svært for organisationer at bibeholde den organiske struktur i takt med at organisationen vokser, hvor det kan blive nødvendigt at indkorpere elementer af den mekanistiske struktur, mens man prøver
at bibeholde det organiske i de divisioner hvor innovation er i højsædet.
\subsection{The Contingency Approach}
Organisationsstrukturen afhænger af den situation, som organisationen befinder sig i.
\\The Contingency Approach er en tilgang til organisationen og dens struktur, hvor man forbereder sig på eventuelle ukendte faktorer, som kan have en indvirkning på organisationen.
\\~\\For at imødekomme denne tilgang til det uforudseelige, er det vigtigt at organisationen på bedst mulig vis designer strukturen således at det giver en form for kontrol over det eksterne miljø,
da mangel på dette vil medføre at organisationen ikke er i kontakt med miljøet, og derfor ikke vil kunne følge med på markeder med en hurtig udvikling.
\\~\\For i sandhed at være en agil organisation, som artiklerne lægger op til, er det derfor vigtigt at man indkorperer en organisk struktur, hvor det giver mening og kan lade sig gøre, samt anvender "The Contingency Approach" løbende for hele tiden at være omstillingsparate ift. markedet.