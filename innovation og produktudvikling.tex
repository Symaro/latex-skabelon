\section{Spørgsmål 1 - Innovation og Produktudvikling}
Denne opgave har til formål at belyse innovation, produktudvikling, og produktudviklingsstrategi.\\
Dette gøres med afsæt i opgivet pensum samt tre af de artikler opgivet i eksamensopgaven, som vil blive
referet til på følgende måde:
\begin{itemize}
    \item \cite[a.1]{eksamensopgave} - "Sådan undgår en virksomhed at forhaste sig på det teknologiske felt"
    \item \cite[a.2]{eksamensopgave} - "Fadøl er det nye våben i et globalt slagsmål"
    \item \cite[a.3]{eksamensopgave} - "Giganterne har rettet sigtekornet mod danske iværksættere"
\end{itemize}

Innovation er den proces, hvorved organisationer anvender deres ekspertise og
ressourcer til at udvikle nye produkter, for bedre at kunne tilpasse de skiftende
behov der måtte være for forbrugerne på det pågældende marked, eller optimere eksisterende metoder for eksisterende produkter.\\
Innovation bliver ofte brugt som et buzzword for signalere nytænkning, men faktum er, at innovation ofte er en ressourcetung proces og op mod 88\% af projekter ikke når ud til brugerne\cite[s. 388]{jones:2013}.\\~\\
Innovation kan opdeles i to typer, "Incremental Technological Change" og "Quantum Technological Change".
\\\paragraph{Incremental Technological Change (ITC)} er teknologi som gradvist bliver forfinet eller udbygget baseret på en allerede
eksisterende teknologi, hvor dette er den mest gængse type af innovation, og ses i vid udstrækning, f.eks. hver gang der lanceres en ny model af en allerede eksisterende smartphone\\
Et eksempel på ITC er at finde i artiklen omkring Plant Jammers \cite[a.3]{eksamensopgave}, hvor Miele
har indkøbt sig i virksomheden, antageligt mod at få et økonomisk afkast eller, måske nok mere sandsynligt, indkorporere teknologien
i en ny række af køleskabe, for at skabe en konkurrencedygtig linje af "smart fridges", for at bl.a. Samsung, LG, Electrolux \cite{fridgemarketshare1, fridgemarketshare2}
\\\paragraph{Quantum Technological Change (QTC)} er de ændringer i teknologien som skaber fundamentale ændringer i enten produkterne, eller måden hvorpå de produceres.
Eksempler på dette er bl.a. udviklingen af den første pc og smartphone, som begge revolutionerede IT-industrien, hvor det i dag er svær at forestille sig en verden
uden disse teknologier. Det kan være svært at spå om på forhånd, hvorvidt en ny teknologi eller tilhørende produktionsmetode, vil have denne revolutionerende effekt, der kan dog, med forbehold,
peges på DraughtMaster fra Carlsberg \cite[a.2]{eksamensopgave}, som et eksempel på dette, dog på en noget mindre skala, end de førnævnte.\\
Grunden til dette er udelukkende pga. den enorme ændring i fustagernes holdbarhed, og derved øger forhandlingsmulighederne for mange restaurationer, som ellers ikke ville
kunne omsætte nok i de givne specialøl, til at opveje omkostningerne.
\\~\\
Som tidligere nævnt er der nævneværdige ressourcer forbundet med innovation, hvilket
også betyder, at timing ift. markedet ofte er alfa og omega, og at det i givne situationer kan betale sig at holde igen, hvilket også belyses i artiklen omkring forhastet innovation \cite[a.1]{eksamensopgave}.
Her påpeges det, at det ofte er first followers som vinder andele på markedet, antageligt fordi de kan udnytte det arbejde som first movers har igangsat og samtidig starte med et produkt, som er tilpasset bedre til det nuværende marked.\\
Derved ikke sagt at det udelukkende er en ulempe at være first mover, da der også er forbundet en række store fordele, såfremt produktet har success. Nedenfor illustreres de fordele og ulemper, som kan være forbundet med dem begge\\
\\\paragraph{First Mover} er innovatørerne bag produktet, og investerer en stor mængde ressourcer, for at komme først på markedet med et nyt produkt.
\begin{itemize}
    \item Fordele
    \begin{itemize}
        \item Mulighed for at oprette patent på det udviklede produkt.
        \item Der er ingen eksisterende konkurrenter af produkttypen på markedet, så kundernes valgmuligheder er begrænset til deres produkt, hvilket også kan medføre en form for standardisering af produktypen fremadrettet.
        \item Organisationen har mere data og erfaring omkring produktet end eventuelle konkurrenter på det pågældende tidspunkt.
    \end{itemize}
    \item Ulemper
    \begin{itemize}
        \item Innovation er en ressourcetung process, og såfremt der ikke udvikles et færdigt produkt, eller udviklingen af denne er for langsom, så er der et potentielt stort økonomisk tab.
        \item Et nyt produkt er ikke garanteret at følge markedet, hvilket også belyses i \cite[a.1]{eksamensopgave}, så der er også en risiko for at produktet ikke passer til markedet på nuværende tidspunkt.
    \end{itemize}
\end{itemize}
\paragraph{First Follower}
\begin{itemize}
    \item Fordele
    \begin{itemize}
        \item Da first movers allerede har investeret en væsentlig mængde ressourcer på udviklingen, er de økonomiske omkostninger for first followers væsentlig lavere.
        \item Det er muligt at vurdere produktet yderligere inden påbegyndelsen af egen udvikling, hvilket reducerer risikoen for at produktet ikke passer til markedet.
    \end{itemize}
    \item Ulemper
    \begin{itemize}
        \item Der er en risiko for at first movers har oprettet patent eller på anden måde beskyttet deres produktet mod nye indtrængere.
        \item Selvom der ikke er forbundet mange ressourcer med den initielle udvikling, som ved first movers, så kan det være nødvendigt at investere en nævneværdig mængde for at sørge for at produktet kommer på markedet inden det bliver fyldt.
    \end{itemize}
\end{itemize}

Roger's adoption curve
Product Life Cycle
Culture for Innovation p403