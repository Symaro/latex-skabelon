\documentclass{article}
\usepackage[utf8]{inputenc}
\usepackage{fancyhdr}
\usepackage[backend=bibtex, style=alphabetic, sorting=ynt]{biblatex}
\usepackage[margin=2.54cm]{geometry}
\usepackage{lastpage}
\usepackage{graphicx}
\usepackage{hyperref}

\newcommand{\quickcharcount}[1]{%
  \immediate\write18{texcount -1 -sum -merge -char -q #1.tex output.bbl > #1-chars.sum }%
  \input{#1-chars.sum} characters (not including spaces)%
}

\renewcommand{\contentsname}{Indholdsfortegnelse}
\renewcommand{\figurename}{Figur}

\hypersetup{
    colorlinks=true,
    linkcolor=blue,
    filecolor=magenta,      
    urlcolor=cyan,
    citecolor=red
}

\addbibresource{bibliography.bib}

\pagestyle{fancy}
\fancyhf{}
\chead{John Laursen}
\rfoot{Side \thepage \hspace{1pt} af \pageref{LastPage}}

\begin{document}
\begin{titlepage}
    \begin{center}
        \vspace*{1cm}
        
        {\huge \textbf{Organisation og Ledelse}}
        \\SI2-KURSUS
        \\\vspace{0.5cm}
        Eksamen E19
        
        \vspace{1.5cm}
        
        \textbf{John Laursen} - jaur13 \\
        
        
        \vspace{15cm}
        
        \hspace*{1.9cm}
        \includegraphics[width=0.3\textwidth]{assets/SDU_BLACK_RGB.png}
        
        Teknisk Fakultet \\
        Syddansk Universitet\\
        13-06-19
    \end{center}
\end{titlepage}
\tableofcontents
\newpage
\section{Spørgsmål 1}
Denne opgave har til formål at belyse innovation, produktudvikling, og produktudviklingsstrategi.\\
Dette gøres med afsæt i opgivet pensum samt tre af de artikler opgivet i eksamensopgaven, som vil blive
referet til på følgende måde:
\begin{itemize}
    \item \cite[a.1]{eksamensopgave} - "Sådan undgår en virksomhed at forhaste sig på det teknologiske felt"
    \item \cite[a.2]{eksamensopgave} - "Fadøl er det nye våben i et globalt slagsmål"
    \item \cite[a.3]{eksamensopgave} - "Giganterne har rettet sigtekornet mod danske iværksættere"
\end{itemize}
\subsection{Innovation}
Innovation er den proces, hvorved organisationer anvender deres ekspertise og
ressourcer til at udvikle nye produkter, for bedre at kunne tilpasse de skiftende
behov der måtte være for forbrugerne på det pågældende marked, eller optimere eksisterende metoder for eksisterende produkter\cite[s. 388]{jones:2013}.\\~\\
Innovation kan opdeles i to typer, "Incremental Technological Change" og "Quantum Technological Change".\\~\\
\textbf{Incremental Technological Change} er teknologi som gradvist bliver forfinet eller udbygget baseret på en allerede
eksisterende teknologi


\textbf{Quantum Technological Change}
fustager?


First Mover / First Follower
Property Rights
Product Life Cycle
Culture for Innovation p403
\input{produktudvikling.tex}
\input{produktudviklingsstrategi.tex}
\clearpage
\section{Spørgsmål 2}

\newpage
\printbibliography[title = {Kilder}]
\end{document}